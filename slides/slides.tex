\documentclass{beamer}
\usepackage[brazilian]{babel}
\usepackage[T1]{fontenc}        % pacote para conj. de caracteres correto
\usepackage{fix-cm} %para funcionar corretamente o tamanho das fontes da capa
\usepackage{color, xcolor}       % pacote para usar fonte Adobe Times e cores
\usepackage[utf8]{inputenc}   % pacote para acentuação
\usepackage{graphicx}  % pacote para importar figuras
\usepackage{amsmath,latexsym,amssymb} %Pacotes matemáticos
\usetheme{Rochester}
\begin{document}
\title{Introdução à Simulação de Circuitos com LTspice}
\author{Renan Birck Pinheiro}
\institute{Universidade Federal de Santa Maria}
\date{\today} 

\frame{\titlepage} 


\begin{frame}
\frametitle{Introdução} 
Por que simular circuitos?
\begin{itemize}
\item Complexidade dos circuitos complica ou inviabiliza a análise manual;
\item Dificuldade ou impossibilidade de prototipagem;
\item Necessidade de agilizar o processo de desenvolvimento.
\end{itemize}
\end{frame}

\begin{frame}
\frametitle{Limitações}
\begin{itemize}
\item Fabricantes \textbf{não fornecem modelos} para os componentes (ou esses são de baixa qualidade);
\item \textbf{Comportamentos não-ideais} (ruído, campos magnéticos...) \textbf{não são modelados};
\item \textbf{Tempo de simulação} pode ser um inconveniente.
\end{itemize}
\end{frame}

\begin{frame}
\frametitle{SPICE}
\begin{itemize}
\item \textbf{S}imulation \textbf{P}rogram with \textbf{I}ntegrated \textbf{C}ircuit \textbf{E}mphasis. Programa de Simulação com Ênfase em Circuitos Integrados;
\item \textbf{Desenvolvido} nos anos 70 para grandes computadores, saída apenas em texto;
\item Código aberto $\rightarrow$ surgiram diversos fabricantes criando novas versões; melhoria dos computadores pessoais permite o uso de gráficos;
\item Usaremos o LTspice por ser gratuito.
\end{itemize}
\end{frame}

\begin{frame}
\frametitle{Instalando o LTspice}
Vá até \url{http://www.linear.com/ltspice} e baixe a versão para Windows (clique em \textit{Download LTspice for Windows} e depois em \textit{No, thanks})\footnote{Há uma versão para Mac disponível, a qual assumo ter as mesmas funções mas não tenho como testar}.
\end{frame}

\begin{frame}
\frametitle{Interface do LTspice}
\end{frame}

\begin{frame}
\frametitle{Desenhando um circuito}
\end{frame}

\begin{frame}
\frametitle{Análise CA (domínio da frequência)}
\end{frame}

\begin{frame}
\frametitle{Análise transiente (domínio do tempo)}
\end{frame}

\begin{frame}
\frametitle{Varredura de parâmetros}
\end{frame}

\begin{frame}
\frametitle{Análise de Fourier}
\end{frame}

\frame{
    \Huge{OBRIGADO!}
}
\end{document}


